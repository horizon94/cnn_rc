\documentclass{article}
\usepackage{fullpage}
\usepackage{times}
\usepackage{cite}
\usepackage{hyperref}
\usepackage{algorithmic}
\usepackage{amssymb}
\usepackage{algorithm}
\usepackage{adjustbox}
\usepackage{graphicx}

\newcommand{\code}[1]{\texttt{#1}}
\begin{document}

\begin{center}
    \vspace{2cm}
    \huge { Thesis Preparation Report \\ }
    \vspace{2cm}
    \large {Magnus Stahl (msta@itu.dk)\\
    IT-University of Copenhagen\\
    Fall 2016}
\end{center}


\section{Introduction}
 
This report details introductory thesis work as part of a spring 2017 thesis at the ITUniversity of Copenhagen. The report will describe the work undertaken so far, and a plan for how the thesis will be completed in the spring. \\

The topic of the thesis is neural networks applied to the task of relation classification. The task of relation classification can be defined as:
Given a sentence $S$ and two labeled entities \textless e1\textgreater  and \textless e2\textgreater , classify the relation between the entities and the (sometimes optionally) ordering of the relation. Here is an example:

Given sentence $S$: \emph{``The train rolled onto the platform with a peculiar choo choo.''}
and entities \textless e1 \textgreater \emph{train}, \textless e2 \textgreater \emph{platform} the relation is of the form Entity-Destination(e1,e2).





\section{Work so far}

\subsection{Motivation}

This section motivates why the subject of the thesis is important and interesting.\\


Classification ... ``Actor dependent classification''

That includes sentence-level understanding of the text to recognize entities and classify their relation. This 

\subsection{Research done}

The material I have investigated are grouped into two broad categories: general educational material about deep learning in NLP, and recent research about neural networks in relation classification in particular,  


Papers read and problems along the way...?


\subsection{Implementation and current baseline}

I'll wait with this until just before the hand-in I guess.


\section{Thesis plan}

\ref{att_cnn}
\cite{att_cnn}
Gant diagram including,
Implementing pairwise margin-based loss function

Attention-based pooling
Contextual embeddings?

Proper baseline description
Construct PROPER table of hyperparameters for ALL LAYERS!!!!!

Investigate named-entity recognition multi-training(?)

Semantic relation stuff, going from general relations to specific
sets (legal documents)

Try cutting off 1 entity and then classifying the relation along with the other entity?


\bibliographystyle{plain}
\bibliography{Bibliography}


    
\end{document}

